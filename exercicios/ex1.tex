\section{}

\subsection{}

\begin{theorem}[Somatório de números naturais]
	Para $n \in \mathbb{N}$, tem-se que $\sum_{k = 1}^n k = \frac{n(n+1)}{2}$.
\end{theorem}

\begin{proof}
	\; \\
	Caso base $n = 1$, tem-se que 1 = 1, pelo que verifica o caso base.\\
	Fixado $n \in \mathbb{N}$, vamos supor:
	\begin{align*}
		\sum_{k = 1}^n k &= \frac{n(n+1)}{2} &&\text{(Hipótese de Indução)}
		\intertext{Pretende-se provar:}
		\sum_{k = 1}^{n + 1} k &= \frac{(n + 1)(n+2)}{2} &&\text{(Tese de Indução)}
		\intertext{Passo de Indução:}
		\sum_{k = 1}^{n + 1} k &= n + 1 + \sum_{k = 1}^n k
		\overset{\text{\tiny passo de indução}}{=}
		n + 1 + \frac{n (n + 1)}{2} \\
							   &= \frac{n (n + 1) + 2(n + 1)}{2}
							   = \frac{(n +2)(n + 1)}{2}
	\end{align*}
	\\
\end{proof}
